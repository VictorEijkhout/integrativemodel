% -*- latex -*-
%%%%%%%%%%%%%%%%%%%%%%%%%%%%%%%%%%%%%%%%%%%%%%%%%%%%%%%%%%%%%%%%
%%%%%%%%%%%%%%%%%%%%%%%%%%%%%%%%%%%%%%%%%%%%%%%%%%%%%%%%%%%%%%%%
%%%%
%%%% This text file is part of the theory writeup on the
%%%% Integrative Model for Parallelism,
%%%% copyright Victor Eijkhout (eijkhout@tacc.utexas.edu) 2014-6
%%%%
%%%% type.tex : master file for report IMP-03
%%%%
%%%%%%%%%%%%%%%%%%%%%%%%%%%%%%%%%%%%%%%%%%%%%%%%%%%%%%%%%%%%%%%%
%%%%%%%%%%%%%%%%%%%%%%%%%%%%%%%%%%%%%%%%%%%%%%%%%%%%%%%%%%%%%%%%
\documentclass[11pt,fleqn,preprint]{impreport}

\taccreportnumber{IMP-03}

\usepackage{geometry,fancyhdr,wrapfig}

\input setup

\title[IMP type system]{The type system of the Integrative Model}
\author[Eijkhout]{Victor Eijkhout\thanks{{\tt
      eijkhout@tacc.utexas.edu}, Texas Advanced Computing Center, The
    University of Texas at Austin}}

\begin{document}
\maketitle

\begin{abstract}
In a previous note we described the mathematical ideas behind 
the Integrative Model for Parallelism.
Here we formalize these as a type system.
This gives us rigorous definitions of all the concepts;
an implementation of IMP can be based on these.
\end{abstract}

\section{Motivation}

In \emph{IMP-01} we described the mathematical theory of the \acf{IMP}.
We are probably not exaggerating in saying that \ac{IMP} is more
based on a mathematical theory
than other systems for parallel programming. 
This also means that its programming concepts are direct translations
of these mathematical concepts. In this note we 
reintroduce the mathematical ideas, but immediately translate them 
into program concepts.

\section{A type system for the IMP model}
\label{sec:formal}
\input formal

%\bibliography{vle}
%\bibliographystyle{plain}

\end{document}
