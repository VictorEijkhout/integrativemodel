% -*- latex -*-
%%%%%%%%%%%%%%%%%%%%%%%%%%%%%%%%%%%%%%%%%%%%%%%%%%%%%%%%%%%%%%%%
%%%%%%%%%%%%%%%%%%%%%%%%%%%%%%%%%%%%%%%%%%%%%%%%%%%%%%%%%%%%%%%%
%%%%
%%%% This text file is part of the theory writeup on the
%%%% Integrative Model for Parallelism,
%%%% copyright Victor Eijkhout (eijkhout@tacc.utexas.edu) 2014-6
%%%%
%%%% shorttype.tex : include file for IMP-11
%%%%
%%%%%%%%%%%%%%%%%%%%%%%%%%%%%%%%%%%%%%%%%%%%%%%%%%%%%%%%%%%%%%%%
%%%%%%%%%%%%%%%%%%%%%%%%%%%%%%%%%%%%%%%%%%%%%%%%%%%%%%%%%%%%%%%%
In \emph{IMP-03} we give a full mathematical definition of the IMP model.
Here we only list some of the basic concepts.

Suppose we have a set of indices~$N$, and subsets $\kw{Ind}=2^N$.
We also have a set~$P$ of processors. Our first major concept then is that
of \indexterm{distribution}: an assignment from processors to subsets
of the indexset:
%
\DataType{Distr}{distributions of data over the processing elements}
         {\kw{Distr}\equiv \kw{Proc} \rightarrow \kw{Ind}}

We concern outselves with the data parallel application of a function:
%
\DataType{Func}{functions with a single output}
  {\kw{Func} \equiv  \kw{Real}^k\rightarrow \kw{Real}}

The signature function is a description of what indices of the input are needed
to compute an index of the output:
%
\DataType{Signature}{Signature of data parallel functions}
         {\kw{Signature}\equiv N\rightarrow \kw{Ind}}

A distributed array is an array plus a distribution on it:
\DataType{DistrArray}{distributed arrays}
    {\kw{DistrArray} \equiv \kw{Array}\circ\kw{Distr} = \kw{Proc}\rightarrow 2^{\kw{Real}}}

We can now define a `kernel' as a function applied between
distributed objects:
%
\DataType{Kernel}{Data parallel functions between distributed arrays}
    {\kw{Kernel}\equiv\kw{Func}\times \kw{DistrArray}\times \kw{DistrArray}.}

An important result is that the beta distribution,
our generalization of the concept of `halo',
is formally constructable in this model:
%
\Function{\beta}{the $\beta$ distribution of a kernel}
    {K=\langle f,x,y\rangle\quad\hbox{then}\quad
    \beta(K) = \sigma_f\bigl( \gamma(K) \bigr)}
%
where $\sigma_f$ is the `signature function' of the data parallel function.

We get to specific execution concept by defining a task,
as a kernel, executed on a specific processor:
%
\DataType
    {Task}
    {A kernel executed on a specific processor}
    {\kw{Task}\equiv\kw{Kernel}\times\kw{Proc}}

We can now define, but will not do so here,
concepts such as messages, or the dependency graph of tasks.

\begin{remark}
 Data parallel functions of more one distributed input can be accomodated
 by having a signature function per input object.
\end{remark}
