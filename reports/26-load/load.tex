% -*- latex -*-
%%%%%%%%%%%%%%%%%%%%%%%%%%%%%%%%%%%%%%%%%%%%%%%%%%%%%%%%%%%%%%%%
%%%%%%%%%%%%%%%%%%%%%%%%%%%%%%%%%%%%%%%%%%%%%%%%%%%%%%%%%%%%%%%%
%%%%
%%%% This text file is part of the theory writeup on the
%%%% Integrative Model for Parallelism,
%%%% copyright Victor Eijkhout 2014-8
%%%% (eijkhout@tacc.utexas.edu)
%%%%
%%%% load.tex : master file for report IMP-26
%%%%
%%%%%%%%%%%%%%%%%%%%%%%%%%%%%%%%%%%%%%%%%%%%%%%%%%%%%%%%%%%%%%%%
%%%%%%%%%%%%%%%%%%%%%%%%%%%%%%%%%%%%%%%%%%%%%%%%%%%%%%%%%%%%%%%%
\documentclass[11pt,fleqn,preprint]{impreport}

\taccreportnumber{IMP-26}

\input setup

\def\bparagraph#1{\paragraph*{\textbf{#1}}}

\title[Load balancing]{Load Balancing in IMP}
\author[Eijkhout]{Victor Eijkhout\thanks{{\tt
      eijkhout@tacc.utexas.edu}, Texas Advanced Computing
    Center, The University of Texas at Austin}}

\begin{document}
\maketitle

\begin{abstract}
 Discussion of load balancing.
\end{abstract}

\acresetall

\section{Theory}

We use the
\emph{diffusion}\index{load!balancing!by diffusion}
model of load balancing~\cite{Cybenko:1989:balancing,HuBlake:diffusion1999}:
\begin{itemize}
\item processes are connected through a graph structure, and
\item in each balancing step they can only move load to their
  immediate neighbours.
\end{itemize}

This is easiest modeled through a directed graph. Let $\ell_i$ be the
load on process~$i$, and $\tau^{(j)}_i$ the transfer of load on an edge
$j\rightarrow i$. Then
\[ \ell_i \leftarrow \ell_i
    + \sum_{j\rightarrow i} \tau^{(j)}_i
    - \sum_{i\rightarrow j} \tau^{(i)}_j
\]
Although we just used a $i,j$ number of edges, in practice
we put a linear numbering the edges. We then get a system
\[ AT=\bar L \]
where
\begin{itemize}
\item $A$ is a matrix of size $|N|\times|E|$ describing what edges
  connect in/out of a node, with elements values equal to $\pm1$ depending;
\item $T$ is the vector of transfers, of size~$|E|$; and
\item $\bar L$ is the load deviation vector, indicating for each node
  how far over/under the average load they are.
\end{itemize}

In the case of a linear processor array this matrix is
under-determined, with fewer edges than processors, but in most cases the
system will be over-determined, with more edges than processes.
Consequently, we solve
%
\[ T= (A^tA)\inv A^t\bar L \qquad\hbox{or} T=A^t(AA^t)\inv \bar L. \]
%
Since $A^tA$ and $AA^t$ are positive indefinite, we could solve the
approximately by relaxation, needing only local knowledge.

\section{Implementation}

Our basic tool is the \n{distribution_sigma_operator}.
Most operators take a \n{multi_indexstruct} and transform it to a new
one; this operator can take the whole distribution into consideration,
which allows us to do things like averaging.

\verbatimsnippet{transform_average}

Unfortunately, this being integer calculation, we lose a couple of
elements, and so we have another operator that stretches a
distribution to a preset size.

\verbatimsnippet{apply_average}

\section{Example}

We test an synthetic benchmark based on
\begin{itemize}
\item a two-dimensional mesh, distributed over a two-dimensional
  processor grid, with
\item work per grid point that is modeled by a time-dependent bell curve
  \[ w(x,t) = 1+e^{-(x-s(t))^2} \quad \hbox{where $s(t)=tN/T$.} \]
\end{itemize}

This model is encountered in practice with, for instance, weather
codes where the adaptive local time integration is strongly
location-dependent.  Our benchmark does not perform actual work: each
processor evaluates the total amount of work for its subdomain and
sleeps for a proportional amount of time.

After each time step we evaluate the total time per processors, and
perform a `diffusion load balancing'
step~\cite{Cybenko:1989:balancing,HuBlake:diffusion1999}. While this,
strictly speaking, optimizes for the previous time step, not the next,
if the load changes slow enough this will still give a performance
improvement.

Indeed, we see up to 20 percent improvement in runtime.

\begin{tabular}{|r|r|r|r|}
  \hline
  Procs:&64&320&672\\
  Balanced runtime:&63&192&856\\
  Unbalanced:&72&266&935\\
  \hline
\end{tabular}

\section{Discussion}

In this report we have shown how the \ac{IMP} programming model makes
it possible to have dynamically changing data distributions, in
particular where the changes are dictated by application demands. We
have given a proof of concept of this by realizing a dynamic load
balancing scheme completely in user space, not relying on any system
software support. Our tests show that the overhead of recomputing
distributions and recreating data does not negate the performance
enhancement of the improved distributions.

\bibliography{vle}
\bibliographystyle{plain}

\end{document}
