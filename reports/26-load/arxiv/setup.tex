% -*- latex -*-
%%%%%%%%%%%%%%%%%%%%%%%%%%%%%%%%%%%%%%%%%%%%%%%%%%%%%%%%%%%%%%%%
%%%%%%%%%%%%%%%%%%%%%%%%%%%%%%%%%%%%%%%%%%%%%%%%%%%%%%%%%%%%%%%%
%%%%
%%%% This text file is part of the theory writeup on the
%%%% Integrative Model for Parallelism,
%%%% copyright Victor Eijkhout (eijkhout@tacc.utexas.edu) 2014-6
%%%%
%%%% setup.tex : packages and macros for the report series
%%%%
%%%%%%%%%%%%%%%%%%%%%%%%%%%%%%%%%%%%%%%%%%%%%%%%%%%%%%%%%%%%%%%%
%%%%%%%%%%%%%%%%%%%%%%%%%%%%%%%%%%%%%%%%%%%%%%%%%%%%%%%%%%%%%%%%

\usepackage{comment}

\usepackage{amsmath,amssymb,graphicx,mdframed,multicol,pslatex,algorithm2e,verbatim,wrapfig}

\includecomment{series}
\excludecomment{notseries}
\includecomment{book}

%%%%%%%%%%%%%%%%\input acdxmacros
%%
%% Acronyms and index
%%

\newif\ifindexmargin \indexmargintrue
\def\marginindex#1{}%{\marginpar{\small\it #1}}
\newcommand{\indexterm}[1]{\emph{#1}\index{#1}%
    \marginindex{#1}}
\newcommand{\indextermtt}[1]{\emph{#1}\index{#1|texttt}%
    \marginindex{#1}}
\let\indextermdef\indexterm
\newcommand{\indextermp}[1]{\emph{#1s}\index{#1}%
    \marginindex{#1}}
\newcommand{\indextermsub}[2]{\emph{#1 #2}\index{#2!#1}%
    \marginindex{#1 #2}}
\newcommand{\indextermsubp}[2]{\emph{#1 #2s}\index{#2!#1}%
    \marginindex{#1 #2s}}
\newcommand{\indextermbus}[2]{\emph{#1 #2}\index{#1!#2}%
    \marginindex{#1 #2}}
\makeatletter
\usepackage{acronym}
\newcommand\indexac[1]{\emph{\ac{#1}}%
  %\tracingmacros=2 \tracingcommands=2
  \edef\tmp{\noexpand\index{%
    \expandafter\expandafter\expandafter
        \@secondoftwo\csname fn@#1\endcsname%
    @\acl{#1} (#1)}}\tmp}
\newcommand\indexacp[1]{\emph{\ac{#1}}%
  %\tracingmacros=2 \tracingcommands=2
  \edef\tmp{\noexpand\index{%
    \expandafter\expandafter\expandafter
        \@secondoftwo\csname fn@#1\endcsname%
    @\acl{#1} (#1)}}\tmp}
\newcommand\indexacf[1]{\emph{\acf{#1}}%
  \edef\tmp{\noexpand\index{%
    \expandafter\expandafter\expandafter
        \@secondoftwo\csname fn@#1\endcsname
    @\acl{#1} (#1)}}\tmp}
\newcommand\indexacs[1]{\emph{\acs{#1}}%
  \edef\tmp{\noexpand\index{%
    \expandafter\expandafter\expandafter
        \@secondoftwo\csname fn@#1\endcsname
    @\acl{#1} (#1)}}\tmp}
\newcommand\indexacstart[1]{%
  \edef\tmp{\noexpand\index{%
    \expandafter\expandafter\expandafter
        \@secondoftwo\csname fn@#1\endcsname
    @\acl{#1} (#1)|(}}\tmp}
\newcommand\indexacstartbf[1]{%
  \edef\tmp{\noexpand\index{%
    \expandafter\expandafter\expandafter
        \@secondoftwo\csname fn@#1\endcsname
    @\acl{#1} (#1)|(textbf}}\tmp}
\newcommand\indexacend[1]{%
  \edef\tmp{\noexpand\index{%
    \expandafter\expandafter\expandafter
        \@secondoftwo\csname fn@#1\endcsname
    @\acl{#1} (#1)|)}}\tmp}
\newcommand\indexacsub[2]{\emph{\ac{#1}}%
  %\tracingmacros=2 \tracingcommands=2
  \edef\tmp{\noexpand\index{%
    \expandafter\expandafter\expandafter
        \@secondoftwo\csname fn@#1\endcsname%
    @\acl{#1} (#1), #2}!#2}\tmp}
\makeatother

\acrodef{API}{Application Programmer Interface}
\acrodef{BFS}{Breadth-First Search}
\acrodef{BH}{Barnes-Hut}
\acrodef{BLAS}{Basic Linear Algebra Subprograms}
\acrodef{BSP}{Bulk Synchronous Parallelism}
\acrodef{CAF}{Co-Array Fortran}
\acrodef{CG}{Conjugate Gradients}
\acrodef{CREW}{Concurrent Read, Exclusive Write}
\acrodef{CSP}{Communication Sequential Processes}
\acrodef{DAG}{Directed Acyclic Graph}
\acrodef{DSL}{Domain-Specific Language}
\acrodef{DTD}{Distributed Termination Detection}
\acrodef{FDM}{Finite Difference Method}
\acrodef{FE}{Finite Element}
\acrodef{FEM}{Finite Element Method}
\acrodef{FMM}{Fast Multipole Method}
\acrodef{FSM}{Finite State Machine}
\acrodef{GA}{Global Arrays}
\acrodef{GPU}{Graphics Processing Unit}
\acrodef{GS}{Gauss-Seidel}
\acrodef{HPC}{High Performance Computing}
\acrodef{HPF}{High Performance Fortran}
\acrodef{IMP}{Integrative Model for Parallelism}
\acrodef{IR}{Intermediate Representation}
\acrodef{ISA}{Incremental Single Assignment}
\acrodef{KNL}{Knights Landing}
\acrodef{MD}{Molecular Dynamics}
\acrodef{MPI}{Message Passing Interface}
\acrodef{NUMA}{Non-Uniform Memory Access}
\acrodef{OO}{Object-Oriented}
\acrodef{ODE}{Ordinary Differential Equation}
\acrodef{PDE}{Partial Differential Equation}
\acrodef{PGAS}{Partitioned Global Address Space}
\acrodef{PRAM}{Parallel Random Access Machine}
\acrodef{RMA}{Remote Memory Access}
\acrodef{RPC}{Remote Procedure Call}
\acrodef{SIMD}{Single Instruction Multiple Data}
\acrodef{SIMT}{Single Instruction Multiple Thread}
\acrodef{SA}{Simulated Annealing}
\acrodef{SMP}{Symmetric Multi-Processor}
\acrodef{SSA}{Static Single Assignment}
\acrodef{SPMD}{Single Program Multiple Data}
\acrodef{SSSP}{Single-Source Shortest Path}
\acrodef{UPC}{Universal Parallel C}

%%%%%%%%%%%%%%%%\input impmacros
%\usepackage[pdftex,colorlinks]{hyperref}
\usepackage{etoolbox}

% refer to the HPSC book
\usepackage{xr-hyper}
\begin{book}
% already loaded through beamer, so only for book
\usepackage[pdftex,colorlinks]{hyperref}
\end{book}

\hypersetup{bookmarksopen=true}
\externaldocument[HPSC-]{../../../../istc/scicompbook/scicompbook}
\newcommand\HPSCref[1]{HPSC-\nobreak\ref{HPSC-#1}}

\hyphenation{exa-scale data-flow}

\def\heading#1{\paragraph*{\textbf{#1}\kern1em}\ignorespaces}

\expandafter\ifx\csname corollary\endcsname\relax
    \newtheorem{corollary}{Corollary}
\fi
\expandafter\ifx\csname lemma\endcsname\relax
    \newtheorem{lemma}{Lemma}
\fi
\expandafter\ifx\csname theorem\endcsname\relax
    \newtheorem{theorem}{Theorem}
\fi
\expandafter\ifx\csname proof\endcsname\relax
 \newenvironment{proof}{\begin{quotation}\small\sl\noindent Proof.\ \ignorespaces}
     {\end{quotation}}
\fi
\expandafter\ifx\csname definition\endcsname\relax
  \newtheorem{definition}{Definition}
\fi
\expandafter\ifx\csname example\endcsname\relax
  \newtheorem{example}{Example}
\fi
\expandafter\ifx\csname remark\endcsname\relax
  \newtheorem{remark}{Remark}
\fi

\def\verbatimsnippetfont{\small}
\def\verbatimsnippet#1{\begingroup\verbatimsnippetfont \verbatiminput{snippets/#1} \endgroup}
\def\quotesnippet#1{\begin{quotation}\includesnippet{#1}\end{quotation}}
\def\includesnippet#1{\input{snippets/#1.tex}}

\def\I{{\cal I}}\def\P{{\cal P}}\def\g{^{(g)}}
\def\RR{{\cal R}}
\def\argmin{\mathop{\mathrm{argmin}}}
\def\mapop{\mathop{\mathtt{map}}}
\def\kw#1{\mathord{\mathrm{#1}}}%{\ifmmode \mathord{\mathrm{#1}} \else $\mathord{\mathrm{#1}}$ \fi}
\def\pred{\mathop{\mathit{pred}}}
\def\succ{\mathop{\mathit{succ}}}
\def\reduce{\mathop{\mathrm{reduce}}}

\def\impref#1{\emph{IMP-#1}~\cite{IMP-#1}}

\usepackage{framed}
\def\DataType#1#2#3{\begin{framed}
  \textbf{Datatype} $\kw{#1}$: #2
  \[ #3 \]
  \end{framed}
}
\def\Function#1#2#3{\begin{framed}
  \textbf{Function} $#1$: #2
  \[ #3 \]
  \end{framed}
}
\def\Functionl#1#2#3#4{\begin{framed}
  \textbf{Function} $#1$: #2
  \begin{equation} #3 \label{#4}\end{equation}
  \end{framed}
}
\usepackage{mdframed}

\def\inv{^{-1}}
\def\card{\mathop{\mathrm{card}}}
\def\CtrlMsg#1{\textsl{#1}}
\def\upcount{\mathop{\uparrow}{}}
\newcommand\diag{\mathbin{\mathrm{diag}}}
\def\take{\mathop{\mathbf 1}}
\def\definedas{\quad\mathrel{:=}\quad}
\def\n{\bgroup\tt\catcode`\_=12 \let\next=}

\def\kernel#1{\left\langle#1\right\rangle}
\def\In{\mathord{\mathrm{In}}}
\def\Out{\mathord{\mathrm{Out}}}
\def\ptop{_{p\rightarrow p}}
\def\ptoq{_{p\rightarrow q}}
\def\qtop{_{q\rightarrow p}}
\def\pqr{_{q\rightarrow p\rightarrow r}}
\def\ptoi{_{p\rightarrow i}}
\def\itop{_{i\rightarrow p}}
\def\Inst{\mathord{\mathrm{Inst}}}
\def\Scat{\mathord{\mathrm{Scat}}}
\def\ScatR{\mathord{\mathrm{ScatR}}}
\def\ScatS{\mathord{\mathrm{ScatS}}}
\def\Thru{\mathord{\mathrm{Thru}}}

\def\Src{\mathord{\mathrm{Src}}}
\def\Tar{\mathord{\mathrm{Tar}}}

\def\twocode {\afterassignment\twocodeb\def\nexta}
\def\twocodeb{\bgroup \catcode`\_=12\relax
              \afterassignment\twocodec\global\def\nextb}
\def\twocodec{\egroup %\show\nexta \show\nextb
  \par\smallskip
  \hskip\unitindent $\vcenter{\hsize=.37\hsize$\nexta$}\quad
   \vcenter{\hsize=.5\hsize\footnotesize\tt\nextb}$
  \par\smallskip
}

%%%%%%%%%%%%%%%%\input inex
\excludecomment{proposal}
\includecomment{longstory}
\includecomment{lrsx}
\includecomment{dataflow}
\excludecomment{public}
\includecomment{implementation}
%%%%\input ../all.inex
%%%%
% no edit
%%%%
\includecomment{paper}

\begin{proposal}
\excludecomment{lrsx}
\excludecomment{paper}
\end{proposal}

\includecomment{insulting}
\includecomment{fullmonty}
\begin{public}
\excludecomment{fullmonty}
\excludecomment{insulting}
\excludecomment{impquestion}
\excludecomment{inprogress}
\end{public}

\begin{fullmonty}
\newenvironment{inprogress}
  {\par\bigskip \mbox{\bf Material in progress}\smallskip}
  {\par\smallskip \mbox{\it End of material in progress}\bigskip}
\newbox\qbox
\newenvironment{impquestion}
  {\begin{quote}\setbox\qbox\vbox\bgroup
                \advance\hsize by -\unitindent {\bf Question}\\}
  {\egroup\fbox{\box\qbox}\end{quote}}
\end{fullmonty}

\includecomment{nonlrsx}
\begin{lrsx}
\excludecomment{nonlrsx}
\end{lrsx}

\includecomment{shortstory}
\begin{longstory}
\excludecomment{shortstory}
\end{longstory}
\includecomment{theory}
\begin{shortstory}
\excludecomment{theory}
\end{shortstory}
%%%%
%%%%

\includecomment{lrsx}
\includecomment{mvpexample}
\includecomment{bh}
\includecomment{programming}
\includecomment{partialdist}
\includecomment{dataflow}

\usepackage{outliner}
\OutlineLevelStart0{\section{#1}}
\OutlineLevelStart1{\subsection{#1}}
\OutlineLevelStart2{\subsubsection{#1}}

\newenvironment{note}{\begin{quotation}Note.}{\end{quotation}}

\def\n{\bgroup\tt\catcode`\_=12 \let\next=}
\def\I{{\cal I}}\def\P{{\cal P}}\def\g{^{(g)}}
\def\bp#1{\mathord{\downarrow_{#1}}}
\def\Bp#1{\mathord{\downarrow\downarrow_{#1}}}

\def\construction{ \\ {\small Unpublished preliminary report. Do not
disseminate.}}
