% -*- latex -*-
%%%%%%%%%%%%%%%%%%%%%%%%%%%%%%%%%%%%%%%%%%%%%%%%%%%%%%%%%%%%%%%%
%%%%%%%%%%%%%%%%%%%%%%%%%%%%%%%%%%%%%%%%%%%%%%%%%%%%%%%%%%%%%%%%
%%%%
%%%% This text file is part of the theory writeup on the
%%%% Integrative Model for Parallelism,
%%%% copyright Victor Eijkhout (eijkhout@tacc.utexas.edu) 2014-7
%%%%
%%%% bigdata.tex : master file for report IMP-27
%%%%
%%%%%%%%%%%%%%%%%%%%%%%%%%%%%%%%%%%%%%%%%%%%%%%%%%%%%%%%%%%%%%%%
%%%%%%%%%%%%%%%%%%%%%%%%%%%%%%%%%%%%%%%%%%%%%%%%%%%%%%%%%%%%%%%%
\documentclass[11pt,fleqn,preprint]{impreport}

\taccreportnumber{IMP-27}

\input setup

\def\bparagraph#1{\paragraph*{\textbf{#1}}}

\title[Data analytics]{Data Analytics in IMP}
\author[Eijkhout]{Victor Eijkhout\thanks{{\tt
      eijkhout@tacc.utexas.edu}, Texas Advanced Computing
    Center, The University of Texas at Austin}}

\begin{document}
\maketitle

\begin{abstract}
  Data analytics in IMP
\end{abstract}

\acresetall

\section{Apache Spark}

Spark is a big data tool that can be described in IMP.
The basic object is an RDD: Resilient Distributed Dataset, which is
analogous to an IMP object.

What is a Seq? What is a Block?

In this report we describe how IMP can cover the expressive
functionality of Spark. We will not go into fault tolerance and such.

\begin{description}
\item[Map] Apply a function to an RDD, giving a new one. We realize
  this by applying the function and letting $\gamma=\alpha$.
\item[FlatMap] Apply a function to return a Seq. We expand the
  distribution accordingly.
\item[MapPartitions] Run a function separately on each block of the partition.
\item[MapPartitionsWithIndex] Run a function separately on each block
  of the partition, and include the index of the partition block.
\item[Filter] Select the elements for which a specified function is
  true. To first order we model this by locally contracting the input
  distribution to the `true' elements.
\item[Union] Combine two datasets. The resulting distribution is
  obvious.
\item[!! Intersection !!] Very tricky! This needs an Allgather or
  so. Better: bucket brigade.
\item[!! Distinct !!] Keep distinct elements. This is global too,
  probably through a bucket brigade of comparisons. Which copy do we
  keep? Lowest location? That may lead to unbalance.
\item[!! GroupByKey !!] Questions: how do we relate the number of keys and
  number of locales? There must be a concept of affinity, but is that
  otherwise visible?
\item[ReduceByKey] Similar.
\item[SortByKey] This is basically sorting. No interaction with
  affinity that we don't already know.
\item[Join] Take $\langle K,V\rangle$ and $\langle K,W\rangle$
  datasets and return $\langle K,(V,W)\rangle$. Just locally blow up:
  affinity of $(V,W)$ is affinity of~$V$. This can of course require
  load balancing.
\end{description}

\section{Clustering algorithms}

See~\cite{IMP-12}.

\section{Minebench}

\index{MineBench}

Data mining benchmark~\cite{MineBench-homepage,MineBench2006}; see table~\ref{tab:minebench}.

The codes as given are OpenMP only.

\begin{table}[ht]
  \begin{tabular}{|l|l|l|}
    \hline
    Application&Category&Description\\
    \hline
    ScalParC&  Classification&
        Decision tree classification\\
    Naive Bayesian&  Classification&
        Simple statistical classifier\\
    K-means&  Clustering&
        Mean-based data partitioning method\\
    Fuzzy K-means&  Clustering&
        Fuzzy logic-based data partitioning method\\
    HOP&  Clustering&
        Density-based grouping method\\
    BIRCH&  Clustering&
        Hierarchical Clustering method\\
    Eclat&  ARM&
        Vertical database, Lattice transversal techniques used\\
    Apriori&  ARM&
        Horizontal database, level-wise mining based on Apriori property\\
    Utility&  ARM&
        Utility-based association rule mining\\
    SNP&  Classification&
        Hill-climbing search method for DNA dependency extraction\\
    GeneNet&  Structure Learning&
        Gene relationship extraction using microarray-based method\\
    SEMPHY&  Structure Learning&
        Gene sequencing using phylogenetic tree-based method\\
    Rsearch&  Classification&
        RNA sequence search using stochastic Context-Free Grammars\\
    SVM-RFE&  Classification&
        Gene expression classifier using recursive feature elimination\\
    PLSA&  Optimization&
        DNA sequence alignment using Smith-Waterman optimization method\\
    \hline
  \end{tabular}
  \caption{Minebench codes}
  \label{tab:minebench}
\end{table}

\begin{description}
\item[K-means] See our report \cite{IMP-12}.
\item[PLSA] For Smith-Waterman, see~\HPSCref{sec:smithwaterman}.
\end{description}
\bibliography{vle,imp}
\bibliographystyle{plain}

\end{document}
