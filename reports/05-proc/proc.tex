% -*- latex -*-
%%%%%%%%%%%%%%%%%%%%%%%%%%%%%%%%%%%%%%%%%%%%%%%%%%%%%%%%%%%%%%%%
%%%%%%%%%%%%%%%%%%%%%%%%%%%%%%%%%%%%%%%%%%%%%%%%%%%%%%%%%%%%%%%%
%%%%
%%%% This text file is part of the theory writeup on the
%%%% Integrative Model for Parallelism,
%%%% copyright Victor Eijkhout (eijkhout@tacc.utexas.edu) 2014-6
%%%%
%%%% proc.tex : master file for report IMP-05
%%%%
%%%%%%%%%%%%%%%%%%%%%%%%%%%%%%%%%%%%%%%%%%%%%%%%%%%%%%%%%%%%%%%%
%%%%%%%%%%%%%%%%%%%%%%%%%%%%%%%%%%%%%%%%%%%%%%%%%%%%%%%%%%%%%%%%
\documentclass[11pt,fleqn,preprint]{impreport}

\taccreportnumber{IMP-05}

\usepackage{geometry,fancyhdr,multirow,wrapfig}

\input setup

\title[IMP processors]{Processors in the Integrative Model}
\author[Eijkhout]{Victor Eijkhout\thanks{{\tt
      eijkhout@tacc.utexas.edu}, Texas Advanced Computing Center, The
    University of Texas at Austin}}

\begin{document}
\maketitle

\begin{abstract}
The IMP model has tasks as the basic units of execution, rather
than processors or processes.
Here we define a process or processor as a collection of tasks,
and we derive some properties of this concept.
\end{abstract}

\section{Motivation}

The \acf{IMP} is based on an execution model in terms
of a dataflow graph\footnote{To be precise, this graph is formally
derived from a description in terms of distributions; see \emph{IMP-01}.}
of tasks, rather than more permanent
notions such as process or processor. This leaves us
with flexibility in how to assign these tasks to actual
physical processors. 
Here we define processors or processes as 
--~not necessarily disjoint~-- subsets of the task graph,
and we derive properties of these processors from
the synchronization mechanisms in \emph{IMP-04}.

\section{Processors and synchronization}
\input execution

\section{Task graphs}
\input taskgraph

\subsection{Clocks}

In distributed processing the question of how to define
a clock often comes up. First formalized by Lamport~\cite{lamport:clock},
a (locally defined) clock is any scalar function $C(\cdot)$
on the events on a process such that `if $a$~causes~$b$, then $C(a)<C(b)$'.
The definition of `causing' is the transitive closure of the
following relations:
\begin{itemize}
\item if $a$ and $b$ are events on the same process,
  and $a$ happens before~$b$, then $a$~is said to cause~$b$;
\item if $a$ is a sending event and $b$~receives a message
  send by~$a$, then $a$~is also said to cause~$b$.
\end{itemize}

In our model it is easy to define a clock satisfying this consistency
criterium. If we identify events with tasks, then an event/task
is uniquely described by a step-domain pair. For the clock,
both locally and globally, we use the step counter.
Both clauses of the causality definition imply that $s(a)<s(b)$,
which is the correct clock ordering.

\section{Task migration}
\input migration

\bibliography{vle}
\bibliographystyle{plain}

\end{document}
