% -*- latex -*-
%%%%%%%%%%%%%%%%%%%%%%%%%%%%%%%%%%%%%%%%%%%%%%%%%%%%%%%%%%%%%%%%
%%%%%%%%%%%%%%%%%%%%%%%%%%%%%%%%%%%%%%%%%%%%%%%%%%%%%%%%%%%%%%%%
%%%%
%%%% This text file is part of the theory writeup on the
%%%% Integrative Model for Parallelism,
%%%% copyright Victor Eijkhout (eijkhout@tacc.utexas.edu) 2014-8
%%%%
%%%% nbody.tex : master file for IMP-09
%%%%
%%%%%%%%%%%%%%%%%%%%%%%%%%%%%%%%%%%%%%%%%%%%%%%%%%%%%%%%%%%%%%%%
%%%%%%%%%%%%%%%%%%%%%%%%%%%%%%%%%%%%%%%%%%%%%%%%%%%%%%%%%%%%%%%%
\documentclass[11pt,fleqn,preprint]{impreport}

\taccreportnumber{IMP-09}

\usepackage{geometry,fancyhdr,multirow,wrapfig,verbatim}

\input setup

\title[IMP nbody]{Tree codes in the Integrative Model}
\author[Eijkhout]{Victor Eijkhout\thanks{{\tt
      eijkhout@tacc.utexas.edu}, Texas Advanced Computing Center, The
    University of Texas at Austin}}

\begin{document}
\maketitle

\begin{abstract}
Many practical algorithms have tree-structured calculations.
In this note we look in particular at N-body algorithms
such as the Barnes-Hut octtree method.
Such calculations have many aspects that are tricky to program
in parallel. We show how the Integrative Model for Parallelism (IMP)
makes their specification more transparent.
\end{abstract}

\section{Background}

\subsection{Barnes-Hut}

\input bhshared

\subsection{Fast Multipole}

See the analysis in~\cite{Yokota:fmm-complexity} for communication
complexity, which takes into account redundant storage of higher tree levels.

\section{Algebraic framework}
\input bhframework

\section{Implementation}

Initially, we consider the tree-structured computation of~$g$.

\input regulartree

\bibliography{vle}
\bibliographystyle{plain}

\end{document}
