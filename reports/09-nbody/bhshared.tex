% -*- latex -*-
%%%%%%%%%%%%%%%%%%%%%%%%%%%%%%%%%%%%%%%%%%%%%%%%%%%%%%%%%%%%%%%%
%%%%%%%%%%%%%%%%%%%%%%%%%%%%%%%%%%%%%%%%%%%%%%%%%%%%%%%%%%%%%%%%
%%%%
%%%% This text file is part of the theory writeup on the
%%%% Integrative Model for Parallelism,
%%%% copyright Victor Eijkhout (eijkhout@tacc.utexas.edu) 2014-8
%%%%
%%%%%%%%%%%%%%%%%%%%%%%%%%%%%%%%%%%%%%%%%%%%%%%%%%%%%%%%%%%%%%%%
%%%%%%%%%%%%%%%%%%%%%%%%%%%%%%%%%%%%%%%%%%%%%%%%%%%%%%%%%%%%%%%%

\begin{wrapfigure}{r}{2in}
  \includegraphics[scale=.3]{nbody_pic}
  \caption{Summing of forces in an N-body problem}
  \label{fig:nbody-sum}
  \vskip-1in
\end{wrapfigure}
%
Algorithms for the $N$-body problem need to compute in each time step
the mutual interaction of each pair out of $N$ particles, giving an
$O(N^2)$ method. 

  \begin{tabbing}
    for \=each particle $i$\\
    \>for \= each particle $j$\\
    \>\> let $\bar r_{ij}$ be the vector between $i$ and $j$;\\
    \>\> then the force on $i$ because of $j$ is\\
    \>\> $\quad f_{ij} = -\bar r_{ij}\frac{m_im_j}{|r_{ij}|}$\\
    \>\> (where $m_i,m_j$ are the masses or charges) and\\
    \>\> $f_{ji}=-f_{ij}$.
  \end{tabbing}

However, by suitable approximation of the `far field'
it becomes possible to have an $O(N\log N)$ or even an $O(N)$
algorithm, see the Barnes-Hut octree method~\cite{BarnesHut} and the
Greengard-Rokhlin fast multipole method~\cite{GreengardRokhlin}.

The naive way of coding these algorithms uses a form where each
particle needs to be able to read values of, in principle, every
cell. 
This is easily implemented with shared memory or an emulation of
it; however, it is difficult to express affinity this way, leading
to potentially inefficient execution.

Pseudo-code would look like:

\begin{verbatim}
Procedure Quad_Tree_Build
    Quad_Tree = {empty}
    for j = 1 to N  // loop over all N particles
         Quad_Tree_Insert(j, root)  // insert particle j in QuadTree
    endfor
    Traverse the Quad_Tree eliminating empty leaves

Procedure Quad_Tree_Insert(j, n) // Try to insert particle j at node n in Quad_Tree
    if n an internal node              // n has 4 children
        determine which child c of node n contains particle j
        Quad_Tree_Insert(j, c)
   else if n contains 1 particle   //  n is a leaf
        add n’s 4 children to the Quad_Tree
        move the particle already in n into the child containing it
        let c be the child of n containing j
        Quad_Tree_Insert(j, c)
    else                                         //  n empty 
        store particle j in node n
    end
\end{verbatim}

\begin{verbatim}
parallel over all particles p:
  cell-list = all top level cells
  sequential over c in cell-list:
    if c is far away, evaluate forces p<->c
    otherwise
      open c and add children cells to cell-list
\end{verbatim}

\begin{verbatim}
// Compute the CM = Center of Mass and TM = Total Mass of all the particles 
( TM, CM ) = Compute_Mass( root )

function ( TM, CM ) = Compute_Mass( n )
  if n contains 1 particle
     store (TM, CM) at n
     return (TM, CM)
  else       // post order traversal
             // process parent after all children
     for all children c(j) of n
           ( TM(j), CM(j) ) = Compute_Mass( c(j) )
     // total mass is the sum
     TM = sum over children j of n: TM(j)
     // center of mass is weighted sum
     CM = sum over children j of n: TM(j)*CM(j) / TM
     store ( TM, CM ) at n
     return ( TM, CM )
\end{verbatim}

\begin{figure}[ht]
  \includegraphics[scale=.08]{bh-quadrants-filled}
  %
  \includegraphics[scale=.08]{bh-quadrants-ratio}
  \caption{Quadrants at size/distance ratio}
  \label{fig:bh-quadrants}
\end{figure}


(This algorithm can also be implemented in distributed memory, with suitable
\emph{latency hiding}\index{latency hiding!in Barnes-Hut}
techniques~\cite{Warren:1993:hash-octree}. However, we prefer
to reformulate it for distributed memory.
Another computational discussion in~\cite{Agullo:pipeFMMinria}.)

