% -*- latex -*-
%%%%%%%%%%%%%%%%%%%%%%%%%%%%%%%%%%%%%%%%%%%%%%%%%%%%%%%%%%%%%%%%
%%%%%%%%%%%%%%%%%%%%%%%%%%%%%%%%%%%%%%%%%%%%%%%%%%%%%%%%%%%%%%%%
%%%%
%%%% This text file is part of the theory writeup on the
%%%% Integrative Model for Parallelism,
%%%% copyright Victor Eijkhout (eijkhout@tacc.utexas.edu) 2014-8
%%%%
%%%%%%%%%%%%%%%%%%%%%%%%%%%%%%%%%%%%%%%%%%%%%%%%%%%%%%%%%%%%%%%%
%%%%%%%%%%%%%%%%%%%%%%%%%%%%%%%%%%%%%%%%%%%%%%%%%%%%%%%%%%%%%%%%

In order for the N-body problem to be efficiently implementable in
the \ac{IMP} model we use a different formulation. In a way this 
is a different arrangement of the statements of the naive formulation.
Such reformulation was already used to arrive at implementations
in distributed memory with message
passing; see~\cite{Salmon86fastparallel}. 

Let us then consider the following form of the N-body algorithms
(see~\cite{Katzenelson:nbody,Agullo:pipeFMMinria}):
\begin{enumerate}
\item\label{fmm:upward} The field due to cell $i$ on level $\ell$ is given by
\begin{equation}
 g(\ell,i) = \oplus_{j\in C^{(\ell)}_i} g(\ell+1,j) 
 \label{eq:BH-g}
\end{equation}
  where $C^{(\ell)}_i$ denotes the set of children of cell~$i$ and
  $\oplus$~stands for a general
  combining operator, for instance computing a joint mass and
  center of mass;
\item\label{fmm:downward} The field felt by cell $i$ on level $\ell$ is given by
\begin{equation}
 f(\ell,i) = f(\ell-1,p(i))+\sum_{j\in S^{(\ell)}_i}g(\ell,j) 
\label{eq:BH-f}
\end{equation}
  where $p(i)$ is the parent cell of~$i$, and $S^{(\ell)}_i$ is the
  interaction region of~$i$: those cells on the same level (`cousins')
  for which we sum the field.
\end{enumerate}

This is a structural description; by suitable choice of kernel it can
correspond to both the \indexterm{Barnes-Hut} or \indexac{FMM}.
%
In \ac{FMM} terms~\cite{Kurzak:FMM}, step~\ref{fmm:upward} is the
`upward pass', by M2M or multipole-to-multipole translation. The
direct calculation of $g(\cdot,\cdot)$ on the finest level is done by
Multipole Expansion Evaluation.
%
The first term in step~\ref{fmm:downward} is the `downward pass' by
L2L or local-to-local translation. The second term is the M2L or
multipole-to-local translation.
