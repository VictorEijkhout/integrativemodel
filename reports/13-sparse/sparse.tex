% -*- latex -*-
%%%%%%%%%%%%%%%%%%%%%%%%%%%%%%%%%%%%%%%%%%%%%%%%%%%%%%%%%%%%%%%%
%%%%%%%%%%%%%%%%%%%%%%%%%%%%%%%%%%%%%%%%%%%%%%%%%%%%%%%%%%%%%%%%
%%%%
%%%% This text file is part of the theory writeup on the
%%%% Integrative Model for Parallelism,
%%%% copyright Victor Eijkhout (eijkhout@tacc.utexas.edu) 2014-6
%%%%
%%%% sparse.tex : master file for report IMP-13
%%%%
%%%%%%%%%%%%%%%%%%%%%%%%%%%%%%%%%%%%%%%%%%%%%%%%%%%%%%%%%%%%%%%%
%%%%%%%%%%%%%%%%%%%%%%%%%%%%%%%%%%%%%%%%%%%%%%%%%%%%%%%%%%%%%%%%
\documentclass[11pt,fleqn,preprint]{impreport}

\taccreportnumber{IMP-13}

\usepackage{geometry,fancyhdr,wrapfig}

\input setup

\title[Sparse operations in IMP]{Sparse Operations in the Integrative Model for Parallelism}
\author[Eijkhout]{Victor Eijkhout\thanks{{\tt
      eijkhout@tacc.utexas.edu}, Texas Advanced Computing Center, The
    University of Texas at Austin}}

\begin{document}
\maketitle

\begin{abstract}
We consider the sparse matrix vector product.
\end{abstract}

\section{Introduction}

In the formal treatment of \acf{IMP} (see IMP-03) we introduced the `indirect'
function $\kw{Ind}$ mapping output indices to required input indices.

%\begin{quotation}
\quotesnippet{indfunction}
%\end{quotation}

As remarked, this function can be represented as a boolean sparse matrix,
and for the case where $f$ computes a matrix-vector product,
that boolean matrix corresponds to the sparsity pattern of~$f$.

\section{Implementation}

We define the sparse matrix-vector product as inheriting from a regular kernel,
by setting the sparse matrix as both the pattern that will determine the beta
distribution, and as context for the local function:
\verbatimsnippet{spmvpkernel}

The pattern is stored as as the local part of the beta distribution:
\verbatimsnippet{pstructfrompattern}

which means that dependency analysis can be done as for an explicit beta:
\verbatimsnippet{analyzepatterndependence}

\section{Remapping and caching}

In the class definition of \n{mpi_spmvp_kernel} above we did not yet
note the overloaded definition of \n{analyze_dependencies}:
this includes a call to remap the matrix:
%
\verbatimsnippet{mpimatrixremap}
%
where MPI remapping is based on an internal routine:
%
\verbatimsnippet{matrixremap}

%\section{Discussion}

%\bibliography{vle}
%\bibliographystyle{plain}

\end{document}
