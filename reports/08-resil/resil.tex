% -*- latex -*-
%%%%%%%%%%%%%%%%%%%%%%%%%%%%%%%%%%%%%%%%%%%%%%%%%%%%%%%%%%%%%%%%
%%%%%%%%%%%%%%%%%%%%%%%%%%%%%%%%%%%%%%%%%%%%%%%%%%%%%%%%%%%%%%%%
%%%%
%%%% This text file is part of the theory writeup on the
%%%% Integrative Model for Parallelism,
%%%% copyright Victor Eijkhout (eijkhout@tacc.utexas.edu) 2014-6
%%%%
%%%% resil.tex : master file for IMP-08
%%%%
%%%%%%%%%%%%%%%%%%%%%%%%%%%%%%%%%%%%%%%%%%%%%%%%%%%%%%%%%%%%%%%%
%%%%%%%%%%%%%%%%%%%%%%%%%%%%%%%%%%%%%%%%%%%%%%%%%%%%%%%%%%%%%%%%
\documentclass[11pt,fleqn,preprint]{taccreport}

\taccreportnumber{IMP-08}

\usepackage{geometry,fancyhdr,multirow,wrapfig}

\input setup

\title[IMP resilience]{Resilience in the Integrative Model}
\author[Eijkhout]{Victor Eijkhout\thanks{{\tt
      eijkhout@tacc.utexas.edu}, Texas Advanced Computing Center, The
    University of Texas at Austin}}

\begin{document}
\maketitle

\begin{abstract}
As supercomputers grow to the exascale, resilience 
--~the ability of a computation to withstand hardware failure~--
is becoming a serious issue.
We outline a solution based on redundant computation.
\end{abstract}

\section{Sketch}

Hardware can fail in multiple ways, from messages arriving out of order, via memory corruption
through cosmic rays, to processors dying. Here we consider the simple, and most disastrous,
case of a  processor completely dropping out of the computation. We assume that the failure
develops in the middle of a computation, rather than during communication. This assumption is warranted
if the network fabric or the communication library has some guarantee that a message is delivered
in total or not at all.

We reason through our solution in the following steps.
\begin{enumerate}
\item Our solution is to duplicate all computation. This carries an obvious factor two overhead cost,
  which we will assume is justifiable.
\item Firstly we note that redundant computation is easily modeled
  in \ac{IMP} through the use of non-disjoint distributions. 
\item Redundant computation also
  brings with it a new complication: every message in the system is also duplicated.
  This means that a receiving process now has two possibilities to fulfill its outstanding receive
  request. \ac{IMP} can accomodate this without further modifications by using tagged sends; 
  see~\cite{IMP-07}.
\item If every task is redundantly executed twice, every message is duplicated, but there will be
  two receivers. 
  \begin{quotation}
    Uh Oh. How do you pair them up? Wildcard receives are easy. Wildcard sends not.
  \end{quotation}
\item The flip side of the tagged send notion is that a process can now have an unmatched outstanding send,
  since the receive operation for the data it is sending may have been satisfied by the duplicate message.
  This is certainly the case if a processor has dropped out.
\end{enumerate}

\section{Failure detection}

The concept of \indexterm{containment domain} was proposed in~\cite{Chung:2012:CDS}. This
requires several user actions, including writing a test function to
decide whether the code in the containment domain has
failed. In~\cite{IMP-04} we showed how the IMP model has the
possibility of detecting failed tasks automatically.


\bibliography{vle}
\bibliographystyle{plain}

\end{document}
