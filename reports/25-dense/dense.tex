% -*- latex -*-
%%%%%%%%%%%%%%%%%%%%%%%%%%%%%%%%%%%%%%%%%%%%%%%%%%%%%%%%%%%%%%%%
%%%%%%%%%%%%%%%%%%%%%%%%%%%%%%%%%%%%%%%%%%%%%%%%%%%%%%%%%%%%%%%%
%%%%
%%%% This text file is part of the theory writeup on the
%%%% Integrative Model for Parallelism,
%%%% copyright Victor Eijkhout (eijkhout@tacc.utexas.edu) 2014-6
%%%%
%%%% dense.tex : master file for report IMP-25
%%%%
%%%%%%%%%%%%%%%%%%%%%%%%%%%%%%%%%%%%%%%%%%%%%%%%%%%%%%%%%%%%%%%%
%%%%%%%%%%%%%%%%%%%%%%%%%%%%%%%%%%%%%%%%%%%%%%%%%%%%%%%%%%%%%%%%
\documentclass[11pt,fleqn,preprint]{impreport}

\taccreportnumber{IMP-25}

\input setup

\def\bparagraph#1{\paragraph*{\textbf{#1}}}

\title[Dense linear algebra]{Dense Linear Algebra in IMP}
\author[Eijkhout]{Victor Eijkhout\thanks{{\tt
      eijkhout@tacc.utexas.edu}, Texas Advanced Computing
    Center, The University of Texas at Austin}}

\begin{document}
\maketitle

\begin{abstract}
  \input abstract
\end{abstract}

\acresetall

\section{Tools for Cartesian parallelism}

The initial \ac{IMP} theory was described in terms of linearized
indices. While it is possible to treat matrices and higher dimensional
objects that way, doing so becomes increasingly forced. Hence we have
implemented tools for multi-dimensional parallelism.

We have the following classes:
\begin{itemize}
\item \n{processor_coordinate}: a multi-dimensional processor
  number. The number of dimensions is a user input.
  \verbatimsnippet{pcoorddim}
\item \n{decomposition}: a Cartesian decomposition of the
  computational domains and their assignment to the processors.
  \verbatimsnippet{decompfromcoord}
\item \n{multi_indexstruct}: for now we assign to each processor a
  structure that is the product of a \n{indexstruct} in each
  dimension.
\item \n{parallel_structure}: description of the \n{multi_indexstruct}
  objects of all processors.
\end{itemize}

\section{Parallel transpose}

Rather than describing the signature function of the operation,
we construct the beta distribution explicitly from the alph:
\verbatimsnippet{transdist}

\bibliography{vle,jdm,morebib}
\bibliographystyle{plain}

\end{document}
