% -*- latex -*-
This project brings together the combined theoretical and practical
interests and expertises of the two PIs, spanning parallel computing
from two complimentary angles.
\begin{description}
\item[Victor Eijkhout] The lead PI has a long history in scientific
  and parallel computing, publishing papers in numerical analysis and
  its parallel realization, as well general papers and books on
  general scientific and parallel computing. He has written parallel software,
  including contributing to the parallel PETSc library, and has been
  engaged in project support at the Texas Advanced Computing Center
  for years.
\item[John McCalpin] has a background in fluid modeling
  (oceanography), and has more than a decade of industry experience in
  microprocessor and parallel architecture design, working at IBM,
  SGI, and AMD. In recent years he has been at the Texas Advanced
  Computing Center, evaluating novel hardware and advising the center
  on the design of clusters.
\end{description}

\begin{comment}
<<<<<<< local
\end{comment}
Thus the PIs have combined active experience in
\begin{itemize}
\item Application development
\item Parallel programming and library development
\item Theory of computing and parallel computing
\item In-depth knowledge of computer architecture.
\end{itemize}
Being both employed at the Texas Advanced Computing Center,
they regularly meet both formally and informally.
Their guidance will instill
on the post-doc and students an especially broad view of parallelism.

The two students will persue complementary research projects,
corresponding to the software and hardware prime research interests of the 
respective PIs. However, they
will be encouraged to interact and publish jointly.
\begin{comment}
=======
The students will pursue complementary research projects,
corresponding to the research interests of the PIs. However, they
will be encouraged to interact and jointly publish.

>>>>>>> other
\end{comment}

Since all participants are at the same institution, we will hold
weekly or bi-weekly separate meetings between students, post-doc, and PIs, 
as the need dictates; and monthly general meetings
with the whole team.

We will
use contacts at national labs and industry to let the students do
summer internships to broaden their perspective on the subject of
parallel programming.

We note that neither PI holds an academic position, and therefore can not
function as Ph.D. advisor as such. However, TACC has good connections
with the Computer Science and Electrical and Computer Engineering departments,
so students will find an primary thesis advisor there, with one of the PIs
functioning as co-advisor. The lead PI has made use of this construct
to full satisfaction of all parties.

\endinput
