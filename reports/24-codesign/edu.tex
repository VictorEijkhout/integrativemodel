\subsection{Post-doc and student involvement}

This project will employ a post-doc and two
graduate students. (While both PIs have a staff appointment,
and thus can not function as sole advisor to a Ph.D. student,
PI Eijkhout has good contacts with the CS department,
where he can co-advise students.)
We define their roles as follows.
\begin{itemize}
\item For the post-doc we will look for a candidate with both a
  extensive knowledge of hardware and experience in parallel programming.
  The post-doc will be required to maintain an overall view
  of the whole project.
\item We will find students with complementary, but overlapping,
  interests in software and hardware.
\end{itemize}
We will use our contacts with industry and national labs to let the 
student do summer internships for broader exposure to 
the practice of current hardware and parallel programming.

During this project we will engage REU (Research Experience for Undergraduates)
students to assist in this work, for instance by doing case-studies of
parallelization using the \ac{IMP} ideas and software.
We will apply for further NSF funding for this.

\subsection{Publication}

We will write technical reports and papers to be published at
conferences and peer-reviewed journals.  The post-doc and students will be
encouraged to write papers on their own, and to present their papers
at conferences.

The interdependence between software and hardware
that we are investigating
is typically underreported in the literature. As by-product of our research
we will develop a unique document that describes these interdependent
issues.

If the programming system makes sufficient progress, we will organize a tutorial
session at the yearly Supercomputing conference.

\subsection{Minority involvement}

We will make every effort to involve women, minorities, and underrepresented groups
when hiring students.

\subsection{Teaching activities}

This project uniquely positions us to produce lecture materials that
approach parallel programming and parallel hardware
from a combined practical/theoretical
viewpoint.

We will produce course materials to be integrated in courses we
  teach in Scientific Computing and Parallel Computing at UT.

\endinput

There is a general perception that parallelism is insufficiently
emphasized in education.  This means that students can learn about an
application area without learning how to parallelize its algorithms;
when they become researchers they are apt to produce code that will
have parallelism as an afterthought and probably suboptimally.  Of
course, this situations is not helped by the fact that there is no one
preferred way of expressing parallelism.

Our work will go some way towards improving this situation. 
The \ac{IMP} model is a natural expression for parallel algorithms
that exposes the relevant aspects of data handling,
so it will be a natural medium for teaching the essentials of parallelism
to people who have no other exposure to parallelism.

