Dr McCalpin is an expert on performance analysis and system
architecture in high performance computing,  with processor and
system design experience at SGI, IBM, and AMD.   McCalpin was
the PI on the recently completed NSF-funded
project ``Design and Implementation of Algorithms for an
Experimental High-Radix Network Switching System'' (NSF CCF-1240652,
\$150k, 2012-10-01 through 2013-09-30), which resulted in 
several papers and technical reports~\cite{McCalpinAsyncMessaging_2014,ASAP2013,MulticoreLAFC_2014,McCalpinLowLevelIO_2014}

The \textbf{intellectual merit} of this project lies in the development of 
algorithms for a high-radix network that provides no ordering guarantees, 
but which eliminates almost all network contention and allows much
higher concurrency in the presence of contention.

The \textbf{broader impact} includes both the enabling of more efficient 
networks for HPC applications and the development of a highly efficient  
communications infrastructure for the widely used Fast Fourier Transform 
algorithm.




\endinput





Dr McCalpin is an expert on performance analysis and system
architecture in high performance computing.  Having recently
returned to academia after 12 years in the computer industry (in
processor and system design), he has no completed results from
recent NSF support.  He is the PI on a recently initiated NSF-funded
project ``Design and Implementation of Algorithms for an
Experimental High-Radix Network Switching System'' (NSF CCF-1240652,
\$150k, 2012-10-01 through 2013-09-30).

The \textbf{intellectual merit} of this project lies in the
development of algorithms for a high-radix network that provides no
ordering guarantees, but also eliminates almost all network
contention.

It is hoped that this will have \textbf{broader impact} by enabling
commercial development of such networks, which are expected to
provide significantly better network throughput for certain classes of
communication-bound problems such as Fast Fourier Transforms and
graph analysis algorithms.

\endinput
Dr. McCalpin is an expert on performance analysis in high performance 
computing, with experience in both academia and in industry. McCalpin
developed and maintains the STREAM benchmark~\cite{STREAM}, and is
a collaborator in the HPC Challenge Benchmark suite~\cite{HPCC}.  

McCalpin led performance analysis in the SGI architecture team for the design 
of the Altix 3000 series of servers.  At IBM, he co-led HPC performance analysis
for POWER4 and POWER5, and led the large-scale architecture team in phase 2
of IBM's DARPA HPCS project.  At AMD, he was chief scientist of the ``Torrenza''
project, working on tightly-coupled accelerators for current and future systems.


\begin{comment}
At SGI, McCalpin was the performance lead on the architecture of the SGI
Altix3000 series of scalable distributed shared memory servers.
At IBM, he co-led HPC performance analysis for the POWER4 systems, 
made major contributions to the POWER5 design, led POWER5 post-silicon
performance analysis, and led the large-scale architecture team in
phase 2 of IBM's DARPA HPCS project ("PERCS").  At AMD, he was the
chief scientist of the "Torrenza" project, enabling third parties to
develop tightly-coupled accelerators for AMD systems and researching
enhanced architectural support for future heterogeneous systems.
\end{comment}

Recently returned to academia from industry, McCalpin is the PI on
a recently initiated NSF-funded project ``Design and Implementation
of Algorithms for an Experimental High-Radix Network Switching System''
(NSF CCF-1240652, \$150k, 2012-10-01 through 2013-09-30).

