Here we give a global overview of our research agenda; a set of detailed research
questions was given in section~\ref{sec:proposed}.

We approach our project from two ends: software and hardware,
but both angles of attack are complementary and will be
persued in close collaboration.

\bparagraph{Management plan}
Since all participants in the project will be located at the same
institution, we will hold monthly general meetings.  The PIs will hold
bi-weekly meetings with the students and post-doc, and the students will
have regular meetings with the post-doc. The students will be encouraged
to collaborate. We will use contacts at national labs and industry to
let the students do summer internships to broaden their perspective on
the subject of parallel programming and parallel hardware.  We are
confident that this project has enough material for two doctoral
degrees and scientific advancement of the post-doc.

\bparagraph{Timeline}
It is hard to divide this project in discrete subprojects, but globally
we will use the following program.
\begin{description}
\item[Year 1] We start by investigating relevant issues in software
  and hardware to determine what semantics can realistically be asked
  to be supported, and how software and hardware can interact.
  Joint investigation of software and hardware will lead to a first overall design.
\item[Year 2] We will investigate the implications of replacing caches
  by local memory, of enhanced DRAM semantics, and of abolishing
  coherence. These are matters related to data transport, and they lead to 
  research questions in software and hardware:
  \begin{itemize}
  \item We will develop the concepts and their practical realization
    in \ac{IMP}, delivering both theory and a prototype
    implementation.
  \item We will construct cycle-accurate simulators of hardware,
    including enhanced DRAM controllers.
  \end{itemize}
\item[Year 3] We will extend our investigations to issues of threading and the synchronization this entails.
  Again, we will develop the theoretical apparatus, an API for expressing it, prototype software,
  and simulators for the hardware designs.
\end{description}

\endinput


\textbf{Software.}
We will investigate parallelizable algorithms for their suitability of
  expression in the \ac{IMP} model. This serves several purposes.
\begin{itemize}
\item First of all, having a repertoire of parallelized algorithms
  will bolster our case for the universality of the model, or
  conversely teach us where its limitations lie.
\item By going through all the details of translating \ac{IMP}
  algorithms down to the hardware we will learn at what levels
  scheduling and optimization can be done: compilation, runtime setup,
  just-in-time, and whether this is done in user-space software,
  system level, or hardware level.
\item As an end result of this software investigation we hope to
  settle on a set of desired primitives for hardware to provide.
\end{itemize}

\textbf{Hardware.}
Approaching the work from the hardware end, we will simulate various
hardware designs and evaluate their performance on the
\ac{IMP}-expressed algorithms.
\begin{itemize}
\item We will start by writing simulators for various aspects of hardware that
  give both cycle-accurate timing, as well as detailed reporting on
  the power budget. 
\item One question to be  investigated is the trade-offs
  between cache and managed local memory by running several algorithms
  through the simulators. Initially we will focus mostly on reducing latency,
  later we will address the power consumption explicitly.
\item Another aspect to be investigated is the balance between sophistication on the
  hardware level versus performance improvements on the algorithm
  level.
\end{itemize}
