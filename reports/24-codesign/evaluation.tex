This project aims to show the feasibility of 
more efficient hardware, and software to exploit it.
Thus we will engage in design, exploration, and simulation.
Neither actual hardware design or the delivery of a full 
software system fits this scope; instead we will 
gauge the success of our research by
\textbf{producing hardware simulators, and prototype software}.

The success of our co-designed software/hardware stack
can be evaluated as follows.

\heading{Programming model capabilities}
The end product of our software research will 
be a vocabulary of concepts expressing
data motion and task synchronization. We will show 
that such concepts can be derived in the framework
of the \ac{IMP} model, both by theoretical analysis
and by proof-of-concept prototype software.

\heading{Hardware semantics}
Our hardware research will show the feasibility 
of efficient hardware that implements
more explicit, and higher level, semantics
than is currently used. We will simulate 
this hardware using existing and newly written simulators.

In particular,
we will deliver detailed computations of the power required
by our design, contrasting this with existing processor designs
at similar effective performance. Co-PI McCalpin
has performed this type of analysis in industry 
as well in recent publications~\cite{ASAP2013,MulticoreLAFC_2014}.
