% -*- latex -*-
The PIs will hire under this project
a post-doc to assist with and to an extent
lead this project. Our emphasis will be on 
letting the post-doc make a successful transition
to becoming an independent researcher. To this
purpose, the post-doc will be enlisted in guiding the students
working on the project; while still being guided by the PIs, the post-doc
will be seen as the PIs' colleague by the students, rather than as a peer.

From the nature of the project, the two students will 
do work of very different nature.
For the post-doc to oversee both student requires
accordingly wide expertise. 
It will be the PIs' job to guide to the post-doc to become
such a simultaneous expert.

\paragraph*{\bf Initial assessment} Upon joining the
project, the PIs will have initial meetings with the post-doc to determine 
extant research goals. Since fresh post-docs may not have given the matter
much thought, the PIs will guide and stimulate 
longer term thinking.

\paragraph*{\bf Milestones and evaluation} In joint discussion between
the PIs and post-doc, milestones for the project will be drawn up
that will be used to gauge the progress of the post-doc's
research. Typically, milestones will be set at half-year points on the
project timeline, to allow for sufficient freedom for 
experimentation, periods of background research, et cetera. On the
shorter term, the post-doc will meet weekly with both
supervisors to discuss immediate progress. The interaction with
students will also be discussed on those occasions. Monthly meeting
with PIs, post-doc, and students will be used to ensure convergence on
common goals. The post-doc will be encouraged to interact informally
outside of planned meetings.

\paragraph*{\bf Stimulation of initiative} The post-doc will be
encouraged to participate in seminars inside UT, and contribute papers
to conferences and workshops outside UT. This will guarantee 
the post-doc's
visibility in the wider community as independent researcher, rather
than as appendages of the PIs' project.

