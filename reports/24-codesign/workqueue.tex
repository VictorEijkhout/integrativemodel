\begin{wrapfigure}{r}{10cm}
  \includegraphics[scale=.4]{graphics/fifo}
  \caption{Plot of processor utilization as function of cycles of work per
    descriptor for different work queue latencies}
  \label{fig:fifo}
\end{wrapfigure}
In section~\ref{sec:issue-thread} we argued that  current
processors cannot efficiently support fine-grained work distribution, 
but there is growing interest in increasing the use of fine-grained 
threading.  Compilers are more likely to generate tasks of some
dozens of cycles than of thousands of cycles. Also naturally
multi-threaded applications are likely to generate many small tasks,
for instance a few hundred cycles in the leaf nodes of an \ac{FMM}
simulation; our programming model is eminently able to 
generate such tasks.

We made a brief simulation of the utilization of a processor
dispatching work items from a FIFO-based workqueue. We assume a 60ns
latency for a hardware based design, and 2000~/ 20,000 cycles for
uncontested~/ contested software implementation. We let the FIFO store
cacheline sized work descriptors, and we plot efficiency as a function
of the number of cycles of work per descriptor.
From figure~\ref{fig:fifo} it is clear that for the anticipated task size
a hardware-based mechanism is a necessity. Such hardware mechanisms are already part
of the processor design, however, they are not accessible to the user.
