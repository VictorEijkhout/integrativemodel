Coherent caches provide an environment in which it is possible to write
programs that communicate across independent threads, but this should
not be confused with an environment that supports communication.
Communication can be made to happen correctly via properly ordered 
sequences of loads and stores, but the communication is indirect, not 
explicit, and cannot be differentiated from local memory traffic uninvolved
with communication.  Since it is not possible to identify which memory references
are associated with communication and which are not, the hardware must implement
a very conservative policy which is not close to optimal for either completely 
independent memory references or for communications-related references.
For example, the hardware can only allow memory references to go a little bit
out of order, so that if a delayed memory reference needs to be ordered, sufficient
information is available to "back up" and ensure that the references appear to 
have happened in order.  For true communication, the hardware does not know 
that stores into shared buffers are intended for a remote process, so they remain
in the producer's cache until requested.  Complex, multi-buffered implementations
of "data" items protected by "flag" items (in separate cache lines), are required to 
prevent excessive cache-to-cache transfers when data items are updated.
Implementation of these operations is difficult and prone to subtle errors -- often 
not discovered until the code is run on a system with a slightly different ordering
model.  In the best cases these communication and synchronization operations
are inefficient on cache-coherent hardware, requiring multiple serialized coherence 
operations.  Generic implementations (focusing on correctness and portability, rather 
than performance), can easily be one to two orders of magnitude slower than a 
hardware implementation designed for the communication semantics required.
