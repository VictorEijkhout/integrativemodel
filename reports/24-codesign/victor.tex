\iparagraph{Relevant project}

Together with Drs. van de Geijn and Yilmaz he has/had projects 
%
(DMS-0625917, `Sparse Direct Solvers' \$500k,
%
OCI 0904907 `High-Fidelity Simulation of Bioelectromagnetic Effects on the Human Body with Petascale Computers' \$1400k)
%
for parallel $hp$-adaptive FEM solvers,
co-advising one Ph.D. thesis~\cite{Kim:2013:dissertation} and has
producing several reports and
papers~\cite{Bientinesi:2010uhm,KimEijkhout:2012Sparse,KimEijkhout:2012GPU,Kim:2013:multilevel}.

The \textbf{intellectual merit} here is firstly the integration of application
knowledge in basic solvers. It illustrates our principles of finding an abstraction
layer that is expressive enough, yet that is based on concepts that allow for 
an efficient execution. Secondly, the intellectual merit lies in the
the development of new parallelization techniques
including tasking and heterogeneous models.

The \textbf{broader impact} of these projects lies to a degree in the application realm
as it will make the use high accuracy FEM available to a larger audience.
Computationally, the impact of these projects lies in the development of techniques
for dealing massive task graphs in heterogeneous environments; these techniques
can be adopted in a variety of other application areas.

\iparagraph{Other funded projects}
%
Dr. Eijkhout is or was involved in the Flame project led by dr. Robert van de Geijn
under NSF projects
%
OCI 1148125 SI2 `A Linear Algebra Software Infrastructure for Sustained Innovation in Computational Chemistry and other Sciences' \$1.2M,
%
CCF 0917167, `Transforming Linear Algebra Libraries', \$500k;
%
OCI 0850750, `Mechanical Transformation of Knowledge', \$200k;
%
CCF 0917096, `Toward Mechanical Derivation of Krylov Methods', \$500k;
%
in the last project he led the development of
a system for deriving Krylov methods~\cite{Eijkhout2010ICCS-krylov}
from basic abstractions.

Dr.\ Eijkhout has explored the application of machine learning
techniques in computational science: NSF-funded projects 
%
(NGS ACI-0203984 `NGS: Component-based Frameworks' \$500k, \$500k, CNS 0406403 `NGS: Self-Adapting Large-scale Solver' \$200k)
%
leading to publications~\cite{Aretal:seamless-sgi,Ba:bomb,Bhowmicketal:application,ibmrd:05sans,DemEtAl:ieeeproc2004,DonEij:iccs2003,EijkFuentes:TOMSmetadata,EijkFuen:architecture,EidDonEij:ipdps2003,EidEijDon:ipdps2004,EijkFuenEidDong:components2005} and a Ph.D. thesis~\cite{Erika:thesis}.

He led an NSF CRI project 
%
(CNS 0751144, `On-Demand Test Problem Server' \$480k)
%
for producing a `test problem server' to be used by
linear algebra researchers.
