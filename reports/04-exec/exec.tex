% -*- latex -*-
%%%%%%%%%%%%%%%%%%%%%%%%%%%%%%%%%%%%%%%%%%%%%%%%%%%%%%%%%%%%%%%%
%%%%%%%%%%%%%%%%%%%%%%%%%%%%%%%%%%%%%%%%%%%%%%%%%%%%%%%%%%%%%%%%
%%%%
%%%% This text file is part of the theory writeup on the
%%%% Integrative Model for Parallelism,
%%%% copyright Victor Eijkhout (eijkhout@tacc.utexas.edu) 2014-6
%%%%
%%%% exec.tex : master file for IMP-04
%%%%
%%%%%%%%%%%%%%%%%%%%%%%%%%%%%%%%%%%%%%%%%%%%%%%%%%%%%%%%%%%%%%%%
%%%%%%%%%%%%%%%%%%%%%%%%%%%%%%%%%%%%%%%%%%%%%%%%%%%%%%%%%%%%%%%%
\documentclass[11pt,fleqn,preprint]{taccreport}

\taccreportnumber{IMP-04}

\usepackage{geometry,fancyhdr,multirow,wrapfig}

\input setup

\title[IMP execution]{Task execution in the Integrative Model}
\author[Eijkhout]{Victor Eijkhout\thanks{{\tt
      eijkhout@tacc.utexas.edu}, Texas Advanced Computing Center, The
    University of Texas at Austin}}

\begin{document}
\maketitle

\begin{abstract}
IMP distributions are defined with respect to abstract processing entities,
leading to a concept of tasks, rather than processes. In this note
we show how processors can be defined, and we describe their interaction
as it arises from the task dataflow.
\end{abstract}

\section{Task execution}
\label{sec:exec}

The \ac{IMP} model leads to a dataflow view of task
execution; see for instance \impref{01}.

\input tasks

\section{Task synchronization in practice}
\input mpiompsync

\section{Implementation}

After the discussion in section~\ref{sec:post-xpct}, this is what task execution
in the \ac{IMP} code looks like. The `post' and `xpct' messages are the actions
of a successor task. SPELL THIS OUT!

\verbatimsnippet{taskexecute}

This connection between tasks is made in \n{algorithm::optimize}:

\verbatimsnippet{queueoptimize}

\begin{itemize}
\item The `lift' commands return the message vector from a task, and zero out
  the container for it on the task. Thus, these lines effect a transfer
  of the message vectors between two tasks.
\item The \n{othertsk} is a task on the same processor, belonging
  to the kernel of the predecessor task. \textbf{We need to prove explicitly that
    this has the right properties.}
\end{itemize}

\bibliography{vle,imp}
\bibliographystyle{plain}

\end{document}
