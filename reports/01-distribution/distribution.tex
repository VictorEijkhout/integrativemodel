% -*- latex -*-
%%%%%%%%%%%%%%%%%%%%%%%%%%%%%%%%%%%%%%%%%%%%%%%%%%%%%%%%%%%%%%%%
%%%%%%%%%%%%%%%%%%%%%%%%%%%%%%%%%%%%%%%%%%%%%%%%%%%%%%%%%%%%%%%%
%%%%
%%%% This text file is part of the theory writeup on the
%%%% Integrative Model for Parallelism,
%%%% copyright Victor Eijkhout (eijkhout@tacc.utexas.edu) 2014-8
%%%%
%%%% distribution.tex : master file for report IMP-01
%%%%
%%%%%%%%%%%%%%%%%%%%%%%%%%%%%%%%%%%%%%%%%%%%%%%%%%%%%%%%%%%%%%%%
%%%%%%%%%%%%%%%%%%%%%%%%%%%%%%%%%%%%%%%%%%%%%%%%%%%%%%%%%%%%%%%%
\documentclass[11pt,fleqn,preprint]{impreport}

\taccreportnumber{IMP-01}

\usepackage{geometry,fancyhdr,verbatim,wrapfig}

\input setup

\title[IMP Distribution theory]{IMP Distribution Theory}
\author[Eijkhout]{Victor Eijkhout\thanks{{\tt
      eijkhout@tacc.utexas.edu}, Texas Advanced Computing Center, The
    University of Texas at Austin}}

\begin{document}
\maketitle

\begin{abstract}
\input abstract
\end{abstract}

\section{Motivation}

The concept of distribution of an indexed object comes up naturally in parallel computing
through the \indexterm{owner computes} rule: the distribution is the
mapping from the processor to the set of indices where it computes the
object. If we now consider the data parallel computation of one object
from another, we have an input distribution and an output
distribution. These are typically specified by the user program.

User distributions can be simple, such as blocked, cyclic, or
block-cyclic, in one or more dimensions. They can also be irregular,
such as they appear in \ac{FEM} calculations. In addition to these
traditional distributions, our theory will cover overlapping and even
fully redundant distributions. We will also cover partial
distributions, for instance as a way to handle adaptive mesh
refinement through our data parallel framework.

\section{Motivating examples}
\label{sec:threepoint-example}
\input threepoint
\subsection{Non-trivial example}

Consider one level of multigrid coarsening, with 6 points divided over
4 processes. Since the coarse level has fewer points than processes,
we need to have some duplication. (The alternative, of having inactive
processes, is harder to code, and in fact less efficient.)

\includegraphics[scale=.12]{nbody-unbalance}

The first two processes achieve their coarsening by a local operation,
but the second pair needs to exchange messages. As we will see, this
non-uniform messaging is easily achieved in \ac{IMP}.

\subsection{Programming the model}
\input programthreepoint

\section{Distributions}
\label{sec:distribution}
\input kerneldistro
\input moredistro

\section{Transformations between distributions}
\label{sec:alpha-beta}
Distributions by themselves are a useful tool, but
their most important application is describing
the communication involved in
converting between one distribution and another. 
As already remarked above, we consider the result 
of an all-gather or the construction of a halo region
as a different distribution of a data set,
rather than as `local data' 
as is mostly done in distributed memory
programming. In this section
we formalize this transformation.

\subsection{Definition}
\label{sec:u-inv-v}
\input transformdef

\input transformation

\section{Signature function}
\label{sec:define-signature}
\index{signature function|(}
\input indirect
\index{signature function|)}

\bibliography{vle,imp}
\bibliographystyle{plain}

\end{document}
