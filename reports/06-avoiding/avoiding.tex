% -*- latex -*-
%%%%%%%%%%%%%%%%%%%%%%%%%%%%%%%%%%%%%%%%%%%%%%%%%%%%%%%%%%%%%%%%
%%%%%%%%%%%%%%%%%%%%%%%%%%%%%%%%%%%%%%%%%%%%%%%%%%%%%%%%%%%%%%%%
%%%%
%%%% This text file is part of the theory writeup on the
%%%% Integrative Model for Parallelism,
%%%%
%%%% copyright Victor Eijkhout 2014-8
%%%% (eijkhout@tacc.utexas.edu)
%%%%
%%%%%%%%%%%%%%%%%%%%%%%%%%%%%%%%%%%%%%%%%%%%%%%%%%%%%%%%%%%%%%%%
%%%%%%%%%%%%%%%%%%%%%%%%%%%%%%%%%%%%%%%%%%%%%%%%%%%%%%%%%%%%%%%%

\label{sec:avoid-framework}

In the third subfigure of figure~\ref{fig:LBs} we showed the
traditional strategy of communication a larger halo than would be
strictly
necessary~\cite{Douglas:caching-multigrid,Eijkhout:poly-smooth,OpJo:improved-ssor}.
With this, and some redundant computation, it is possible to remove a
synchronization point from the computation.

However, this is not guaranteed to overlap communication and
computation; also, it is possible to avoid some of the redundant work.
We will now formalize this `communication avoiding'
strategy~\cite{Demmel2008IEEE:avoiding}.

Let $L_{k,p}$ be a collection of local computations. We assume that it
is a well-formed collection in the sense of
definition~\ref{def:localcomp}. We will now split $k$ in three
substeps, $k_1,k_2,k_3$, giving us a splitting that is well-formed,
and that has overlap of computation and communication.

\begin{figure}[ht]
  \includegraphics[scale=.4]{L123}
  \caption{Subdivision of a local computation for minimizing communication and redundant computation.}
  \label{fig:avoid}
\end{figure}

\heading{Subset 0: inherited from previous level}

In step~$k$ we can assume that data from a previous step is
available. This can be either a initial condition, or an actually
computed step. We define $L^{(0)}_{k,p}$ as the data that is available
on process~$p$ in step~$k$ before any computation takes place in step~$k$:
\[
    L^{(0)}_{k,p} \equiv L_{k-1,p}
\]

\heading{Subset 4: all locally computed tasks}

Starting with $L^{(0)}_{k,p}$, some tasks in $L_{k,p}$ can be computed without
needed data from processors $q\not=p$. We term this $L^{(4)}_{k,p}$,
but note that this set is an auxiliary in our story; it is not
computed as such.
\[
    L^{(4)}_{k,p} \equiv
    \bigl\{ t\colon \pred(t)\in \{ L^{(0)}_{k,p} \cup L^{(4)}_{k,p} \} 
    \bigr\}
\]
(This definition is implicit, but well-formed; it is a stand-in for a
more complicated explicit one. We hope the reader bears with us.)

\heading{Subset 5: all predecessors of local tasks}

Next we define $L^{(5)}_{k,p}$ as all tasks in $L_k$
that are computed anywhere to construct the local result~$L_{k,p}$:
\[
    L^{(5)}_{k,p} \equiv
    L_{k,p} \bigcup \bigl( \pred(L_{k,p})\cup L_k \bigr)
\]
Loosely, this contains all local tasks and the remote ones in the
extended halo. Again, this set is an auxiliary, used to defined the
following actually computed steps.

\heading{Subset 1: locally computed tasks, to be used remotely}

With the local tasks $L^{(4)}_{k,\star}$ and the needed tasks
$L^{(5)}_{k,\star}$, we now define $L^{(1)}_{k,p}$ as the locally
computed tasks on~$p$ that are needed for a $q\not=p$:
\[
    L^{(1)}_{k,p}\equiv
    L^{(4)}_{k,p} \cup \bigcup_{q\not=p} L^{(5)}_{k,q}
    - L^{(0)}_{k,p}
\]
These are the tasks computed first. For each $q\not=p$, a subset of
these elements will be sent to process~$q$, in a communication step
that overlaps the computation of the next subset.

\heading{Subset 2: locally computed tasks, only used locally}

While elements of $L^{(1)}_{k,p}$ are being sent, we can do a local
computation of the remainder of~$L^{(4)}_{k,p}$:
\[
    L^{(2)}_{k,p} \equiv L^{(4)}_{k,p} - L^{(1)}_{k,p}
\]

\heading{Subset 3: halo elements and their successors}

Having received remote elements $L^{(1)}_{k,q}$ from neighbouring
processors~$q\not=p$, we can construct the remaining elements
of~$L^{(5)}_{k,p}$ that are needed for~$L_{k,p}$:
\[
    L^{(3)}_{k,p} \equiv
    L^{(5)}_{k,p} - L^{(4)}_{k,p} - \bigcup_{q\not=p} L^{(1)}_{k,q}
\]

We omit definition of the precise elements of~$L^{(1)}_{k,\star}$ that
are transported; we merely note that this set is considerably smaller
than $L^{(1)}_{k,\star}$ itself.

\begin{theorem}
  The splitting $L_{k_1,p},L_{k_2,p},L_{k_3,p}$ is well-formed and has overlap
  of communication $L_{k_1}\rightarrow L_{k_3}$ with the computation of $L_{k_2}$.
  Neither $L_{k_1}$ nor $L_{k_2}$ have synchronization points, so the whole algorithm
  has overlap.
\end{theorem}

However, note that $L_{k_1,p}\cup L_{k_2,p}\cup L_{k_3,p}$ is most
likely larger than~$L_{k,p}$,
corresponding to redundant calculation.

%\textbf{Write out predecessor relations with proof}

\begin{figure}[ht]
  \includegraphics[scale=.4]{L123456}
  \caption{Communicated sets in the communication avoiding scheme}
  \label{fig:avoid-comm}
\end{figure}

In figure~\ref{fig:avoid-comm} we indicate in red the part of
$L^{(0)}$ that is sent, and the part of~$L^{(3)}$ that is received.
