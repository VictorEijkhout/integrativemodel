% -*- latex -*-
%%%%%%%%%%%%%%%%%%%%%%%%%%%%%%%%%%%%%%%%%%%%%%%%%%%%%%%%%%%%%%%%
%%%%%%%%%%%%%%%%%%%%%%%%%%%%%%%%%%%%%%%%%%%%%%%%%%%%%%%%%%%%%%%%
%%%%
%%%% This text file is part of the theory writeup on the
%%%% Integrative Model for Parallelism,
%%%% copyright Victor Eijkhout (eijkhout@tacc.utexas.edu) 2014-6
%%%%
%%%%%%%%%%%%%%%%%%%%%%%%%%%%%%%%%%%%%%%%%%%%%%%%%%%%%%%%%%%%%%%%
%%%%%%%%%%%%%%%%%%%%%%%%%%%%%%%%%%%%%%%%%%%%%%%%%%%%%%%%%%%%%%%%

Let's revisit the example of a one-dimensional \indexterm{heat
  equation} that we started this story with. (This will also stand
for a single sparse matrix-vector product.)
%
\begin{figure}[ht]
  \includegraphics[scale=.12]{div-avoid-spmvp}
  \caption{Split computation of the sparse matrix-vector product, in
    our communication avoiding framework.}
  \label{fig:avoidmvp}
\end{figure}
%
In figure~\ref{fig:avoidmvp} we overlay one step of a heat equation
on the previous definition of a communication avoiding scheme.
However, since we need to block kernels to be able to derive the
$k_{1,2,3}$ regions, we do the following:
\begin{enumerate}
\item We introduce a first kernel which establishes the beta
  distribution of the heat update~/ matrix vector product. In IMP
  terms this is a copy operation between $\alpha_{\mathrm{spmvp}}$ and
  $\beta_{\mathrm{spmvp}}$.
\item We let the actual heat update~/ matrix-vector product go between
  $\beta_{\mathrm{spmvp}}$ as input distribution and
  $\gamma_{\mathrm{spmvp}}$ as output. This is a local operation!
\end{enumerate}
Now we have two blocked kernels and we can analyze a communication
avoiding scheme as described above.
We find that:
\begin{enumerate}
\item $k_1$ are the points that need to be sent as halo
  data. Computing these is of course trivial: we already have them.
\item Next we send off our $k_1$ elements, and post a receive for the
  ones from neighbour processors.
\item Then we compute $k_2$, the local part of the update/product.
\item After receiving the remote $k_1$ sets we can compute the missing
  parts of the update/product.
\end{enumerate}

Of course, this analysis is easy to do in the abstract. In code it
would require that each point of the vector-to-be-updated is a single
task. Alternatively, we need a code splitting mechanism that is
outside of the scope of our work for now. We briefly address this
question in~\cite{IMP-16}.
