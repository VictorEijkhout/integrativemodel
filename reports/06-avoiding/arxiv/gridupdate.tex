% -*- latex -*-
%%%%%%%%%%%%%%%%%%%%%%%%%%%%%%%%%%%%%%%%%%%%%%%%%%%%%%%%%%%%%%%%
%%%%%%%%%%%%%%%%%%%%%%%%%%%%%%%%%%%%%%%%%%%%%%%%%%%%%%%%%%%%%%%%
%%%%
%%%% This text file is part of the theory writeup on the
%%%% Integrative Model for Parallelism,
%%%% copyright Victor Eijkhout (eijkhout@tacc.utexas.edu) 2014-6
%%%%
%%%% gridupdate.tex : include file for IMP-06
%%%%
%%%%%%%%%%%%%%%%%%%%%%%%%%%%%%%%%%%%%%%%%%%%%%%%%%%%%%%%%%%%%%%%
%%%%%%%%%%%%%%%%%%%%%%%%%%%%%%%%%%%%%%%%%%%%%%%%%%%%%%%%%%%%%%%%

In \ac{PDE} methods, a repeated sequence of sparse matrix-vector products
is a regular occurrence.
Typically, the sparse matrix can best be viewed as an operator
on a grid of unknowns, where a new value is some combination of values
of neighbouring unknowns.
In a parallel context this means that in order to evaluate the matrix-vector
product $y\leftarrow Ax$ on a processor, that processor needs to obtain the $x$-values
of its \indexterm{ghost region}. Under reasonable assumptions on the partitioning
of the domain over the processors, the number of messages involved will be fairly
small:
in a \ac{FEM} or \ac{FDM} context,
the number of messages is $O(1)$ as~$h\downarrow\nobreak 0$.

Since there is little data reuse, and in the sparse case not even 
spatial locality, it is normally concluded that the sparse
product is largely a \emph{bandwidth-bound algorithm}. 
Looking at just a
single product there is not much we can do about that. 
However, 
if a number of such products is performed in a row, for instance as the steps
in a time-dependent process, there may be rearrangements
of the operations that lessen the bandwidth demands, typically by lessening the
latency cost.

Consider as a simple example
\begin{equation}
\forall_i\colon x^{(n+1)}_i = f\bigl( x^{(n)}_i, x^{(n)}_{i-1}, x^{(n)}_{i+1} \bigr)
\label{eq:3p-average}
\end{equation}
and let's assume that the set $\{x^{(n)}_i\}_i$ is too large to fit 
in cache.
This is a model for, for instance, the explicit scheme for the heat
equation in one space dimension.
In the ordinary computation, where we first compute all~$x^{(n+1)}_i$, 
then all~$x^{(n+2)}_i$, the intermediate values at level~$n+1$
will be flushed from the cache
after they were generated, and then brought back into cache as input for the
level $n+2$ quantities.

However,
if we compute not one, but two iterations, the intermediate values
may stay in cache.
Consider $x^{(n+2)}_0$: it requires $x^{(n+1)}_0,x^{(n+1)}_1$,
which in turn require $x^{(n)}_0,\ldots,x^{(n)}_2$.

Now suppose that we are not interested in the intermediate results, but
only the final iteration. Figure~\ref{fig:grid-update-overlap} shows
a simple example.
\begin{figure}[ht]
\includegraphics[scale=.1]{grid-update-overlap}
\caption{Computation of blocks of grid points over multiple iterations}
\label{fig:grid-update-overlap}
\end{figure}
The first processor computes 4~points on level $n+2$. For this it needs 5~points
from level $n+1$, and these need to be computed too, from 6~points on level~$n$.
We see that a processor apparently needs to collect a \indexterm{ghost region}
of width two, as opposed to just one for the regular single step update.
One of the points computed by the first processor is $x^{(n+2)}_3$,
which needs $x^{(n+1)}_4$. This point is also needed for the computation
of $x^{(n+2)}_4$, which belongs to the second processor.

The easiest solution is to let this sort of point on the intermediate
level \emph{redundantly computed}\index{redundant computation}, in 
the computation of both blocks where it is needed, on two different processors.

\begin{itemize}
\item First of all, as we motivated above, doing this 
on a single processor increases locality: if all points in a coloured block
(see the figure) fit in cache, we get reuse of the intermediate points.
\item Secondly, if we consider this as a scheme for distributed memory computation,
it reduces message traffic. Normally, for every update step the processors
need to exchange their boundary data. If we accept some redundant duplication
of work, we can now eliminate the data exchange for the intermediate levels.
The decrease in communication will typically outweigh the increase in work.
\end{itemize}

\Level 1 {Analysis}

Let's analyze the algorithm we have just sketched.  As in
equation~\eqref{eq:3p-average} we limit ourselves to a 1D set of
points and a function of three points. The parameters describing the
problem are these:
\begin{itemize}
\item $N$ is the number of points to be updated, and $M$~denotes the
  number of update steps. Thus, we perform $MN$ function evaluations.
\item $\alpha,\beta,\gamma$ are the usual parameters describing
  latency, transmission time of a single point, and time for an
  operation (here taken to be an $f$ evaluation).
\item $b$ is the number of steps we block together.
\end{itemize}
Each halo communication consists of $b$ points, and we do this $\sqrt
N/b$ many times.  The work performed consists of the $MN/p$ local
updates, plus the redundant work because of the halo. The latter term
consists of $b^2/2$ operations, performed both on the left and right
side of the processor domain.

Adding all these terms together, we find a cost of
\[ \frac Mb\alpha+M\beta+\left(\frac {MN}p+Mb\right)\gamma. \]
We observe that the overhead of $\alpha M/b+\gamma Mb$ is independent of~$p$,
Note that  the optimal value of~$b$ only depends on
  the architectural parameters $\alpha,\beta,\gamma$ but not on the
  problem parameters.

\Level 1 {Communication and work minimizing strategy}

We can make this algorithm more efficient by overlapping the
communication and computation. As illustrated in
figure~\ref{fig:grid-update-local}, each processor start by
communicating its halo, and overlapping this communication with the
part of the communication that can be done locally. The values that
depend on the halo will then be computed last.

\begin{figure}[ht]
\includegraphics[scale=.1]{grid-update-local}
\caption{Computation of blocks of grid points over multiple iterations}
\label{fig:grid-update-local}
\end{figure}

If the number of points per processor is large enough, the amount of
communication is low relative to the computation, and you could take
$b$ fairly large. However, these grid updates are mostly used in
iterative methods such as the \indexac{CG} method, and in that case
considerations of roundoff prevent you from taking $b$ too
large\cite{ChGe:sstep}.

A further refinement of the above algorithm is possible.
Figure~\ref{fig:grid-update-minimal} illustrates that it is possible
to use a halo region that uses different points from different time steps.
\begin{figure}[ht]
\includegraphics[scale=.1]{grid-update-minimal}
\caption{Computation of blocks of grid points over multiple iterations}
\label{fig:grid-update-minimal}
\end{figure}
This algorithm (see~\cite{Demmel2008IEEE:avoiding}) cuts down on the amount
of redundant computation. However, now the halo values that are communicated
first need to be computed, so this requires splitting the local communication
into two phases.

